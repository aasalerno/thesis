\chapter{Introduction}

    \section{Magnetic Resonance Imaging}
    
        \subsection{Clinical MRI}
            
            In Ontario, the current wait time to receive a diagnostic MRI scan for 90\% of adult patients is 104 days and 105 days for pediatric patients. The government target for both of these demographics are wait times of 28 days, making the actual value just under 4 times greater than the goal. For life threatening conditions, patients are often pushed to the front of the line. \textbf{CITATION - govt}. Canada also falls below the median number of 6.1 MRI scanners per million people for all of the countries within the Organisation for Economic Co-operation and Development with only 4.6 \cite{Emery09,Stein05}.
            
            MRI scans are a preferred method of scanning due to their ability to visualize structures within the body, being minimally invasive with no ionizing radiation, and having good tissue contrast. The benefits of MRI are also present in the multiple methods of weighting a scan of the body to be more or less sensitive to different tissues within the body. However, for all of these benefits, the major detriment of MRI is the amount of time that it takes to run a scan in comparison to other methods of imaging such as ultrasound and CT. \textbf{CITATION AND INFORMATION}.
    
            Currently, MRI is used as a common diagnostic tool in the clinic -- with the average scan lasting from \textbf{BLANK TO BLANK, CITATION}. The use of MRI is diverse because of it's high sensitivity, which is especially apparent for the location 
            
            
        \subsection{Pre-Clinical MRI}
        
        \subsection{Mouse Models}
        
    \section{Compressed Sensing}
        
        \subsection{Compressed Sensing Theory}
        
        \subsection{Current Clinical Use}
        
        \subsection{}